
\documentclass{article}%
\usepackage{cfr-lm}%
\usepackage[T1]{fontenc}%
% \usepackage[utf8]{inputenc}%
\usepackage{physics}
\usepackage{amsmath}
\usepackage{amssymb}
\usepackage{graphicx}
\usepackage[margin=3cm]{geometry}
\usepackage{changepage}
\usepackage{fontspec}
\usepackage{minted}
\usepackage{tcolorbox}
\usepackage{lmodern}
\usemintedstyle{bw}
\usepackage[%  
    colorlinks=true,
    pdfborder={0 0 0},
    linkcolor=red
]{hyperref}

\usepackage{xcolor}
% \usepackage{tikz}



\newtcbox{\entoure}[1][black]{on line,
arc=2pt,colback=#1!10!white,colframe=white,
before upper={\rule[-3pt]{0pt}{10pt}},boxrule=1pt,
boxsep=0pt,left=2pt,right=2pt,top=1pt,bottom=.5pt}



\title{MAT1120\\Oblig 2}%
\author{Brage Wiseth}%
\date{\today}%
\newfontfamily{\iosevka}{Iosevka Fixed Curly}%
% \setmainfont{FreeSans}
\setmonofont{Iosevka Fixed Curly}
%
\begin{document}%
\normalsize%
% \iosevka%
%
\maketitle%
% \normalfont

\begin{tcolorbox}[colframe=white]
    \textbf{Oppgave I}
\end{tcolorbox}
$A =
\begin{pmatrix}
    0           &   0           &   \frac{1}{2} &   1\\[6pt]
    \frac{1}{3} &   0           &   0           &   0\\[6pt]
    \frac{1}{3} &   \frac{1}{2} &   0           &   0\\[6pt]
    \frac{1}{3} &   \frac{1}{2} &   \frac{1}{2} &   0
\end{pmatrix}
\qquad A - I = 
\begin{pmatrix}
    -1          &   0           &   \frac{1}{2} &   1\\[6pt]
    \frac{1}{3} &  -1           &   0           &   0\\[6pt]
    \frac{1}{3} &   \frac{1}{2} &  -1           &   0\\[6pt]
    \frac{1}{3} &   \frac{1}{2} &   \frac{1}{2} &  -1
\end{pmatrix}
\sim 
\begin{pmatrix}
    1    &   0  &   0   &   -\frac{4}{3}\\[6pt]
    0    &   1  &   0   &   -\frac{4}{9}\\[6pt]
    0    &   0  &   1   &   -\frac{2}{3}\\[6pt]
    0    &   0  &   0   &   0
\end{pmatrix}
$\vspace*{2mm}\\
Det gir løsningen $\vb x = x_2\begin{bmatrix}
    \frac{4}{3}\\[6pt] \frac{4}{9}\\[6pt] \frac{2}{3}\\[6pt] 1
\end{bmatrix} $
til likningen $(A-I)\vb x = 0$ Der $\begin{bmatrix}
    \frac{4}{3}\\[6pt] \frac{4}{9}\\[6pt] \frac{2}{3}\\[6pt] 1
\end{bmatrix} 
$ er en basis for nullrommet til $A$. Score-vektoren blir da basisvektoren normalisert
$\begin{bmatrix}
    \frac{12}{31}\\[6pt] \frac{4}{31}\\[6pt] \frac{6}{31}\\[6pt] \frac{9}{31}
\end{bmatrix}$\vspace*{2mm}\\
En fornuftig rangering av sidene kan da være følgende:
\begin{center}
\begin{tcolorbox}[halign title=center, box align=top, width=.22\linewidth]  
    \begin{enumerate}
        \item Side 1
        \item Side 4
        \item Side 3
        \item Side 2
    \end{enumerate}
\end{tcolorbox}
\newpage
\end{center}
\begin{tcolorbox}[colframe=white]
    \textbf{Oppgave II}
\end{tcolorbox}
$A=
\begin{pmatrix}
    0 &   0   &   0             &   0   &   1\\
    0 &   0   &   \frac{1}{2}   &   1   &   0\\
    0 &   0   &   0             &   0   &   0\\
    0 &   1   &   \frac{1}{2}   &   0   &   0\\
    1 &   0   &   0             &   0   &   0
\end{pmatrix}
\qquad A-I=
\begin{pmatrix}
   -1 &   0   &   0             &   0   &   1\\
    0 &  -1   &   \frac{1}{2}   &   1   &   0\\
    0 &   0   &  -1             &   0   &   0\\
    0 &   1   &   \frac{1}{2}   &  -1   &   0\\
    1 &   0   &   0             &   0   &  -1
\end{pmatrix}
\sim
\begin{pmatrix}
     1 &   0   &   0   &   0   &  -1\\
     0 &   1   &   0   &  -1   &   0\\
     0 &   0   &   1   &   0   &   0\\
     0 &   0   &   0   &   0   &   0\\
     0 &   0   &   0   &   0   &   0
 \end{pmatrix}
$\vspace*{2mm}\\
Det gir løsningen $\vb x = x_4 
\begin{bmatrix}
    0\\ 1\\  0 \\ 1 \\  0
\end{bmatrix}+
x_5
\begin{bmatrix}
    1 \\ 0 \\ 0\\ 0 \\ 1
\end{bmatrix} $ 
til likningen $(A-I)\vb x = 0$ Der 
$
\begin{bmatrix}
    0\\ 1\\  0 \\ 1 \\  0
\end{bmatrix},
\begin{bmatrix}
    1 \\ 0 \\ 0\\ 0 \\ 1
\end{bmatrix} $
er en basis for nullrommet til $A$. Vi har nå funnet to score-vektorer $\begin{bmatrix}
    0\\ \frac{1}{2}\\  0 \\ \frac{1}{2} \\  0
\end{bmatrix},
\begin{bmatrix}
    \frac{1}{2} \\ 0 \\ 0\\ 0 \\ \frac{1}{2}
\end{bmatrix} $ Det kan tolkes som steady state for hver av de to usammenhengede nettverkene.
\begin{tcolorbox}[colframe=white]
    \textbf{Oppgave III}
\end{tcolorbox}
Begge link-matrisene fra oppgave 1 og 2 er stokastiske. De er kvadratiske, 
summen av koefisientene i hver kolonne er 1, og alle innganger er positive og mindre eller lik 1.
Om de er regulære er en annen sak. Hvis vi gjennomfører kalkulasjonen $A^k$ i MatLab på matrisen
fra oppgave 1 finner vi at $A^4$ gir en matrise med strengt positive koefisienter. Matrisen
fra oppgave 1 er dermed regulær. Hvis vi tar potensen fra 1 til $k$ av matrisen fra oppgave 2 finner vi at 
når $k=2$ har matrisen returnert til sin opprinnelige form uten at foregående potenser har 
resultert i strengt positive koefisienter. Matrisen fra oppgave 2 kan dermed ikke være regulær.
\begin{tcolorbox}[halign title=center, box align=top] 
En link-matrise som ikke er stokastisk kan for eksempel oppstå hvis et dokument 
ikke har noen hyperlenker. Det vil tilsvare en kolonne med bare nuller. Vi kan også mistenke at 
ett usammenhengede nettverk kan gi en ikke-regulær matrise.
\end{tcolorbox}
\begin{tcolorbox}[colframe=white]
    \textbf{Oppgave IV}
\end{tcolorbox}
$$M = (1-m)A + mS$$
Summen av koefisientene i en kolonne i $S$ vil bli 1 siden koefisientene er $\frac{1}{n}$ delt opp i $n$
deler, summen er da det samme som $\frac{1}{n} n = 1$. $A$ er en stokastisk matrise så summen av koefisientene
i en kolonne er også 1. Summen av koefisientene i en kolonne $M_i$ i $M$ kan skrives som
$$\text{kolonnesum}(M_i) = (1-m)\sum_{j=0}^n a_{i,j} + m\sum_{j=0}^n s_{i,j} $$
Vi vet at summen av koefisientene i både $A$ og $S$ er 1 $\leadsto \text{kolonnesum}(M_i) = (1-m) + m = 1$.\\
Dette sammen med at $0<m<1$ som sørger for at alle koefisienter i $M$ er positive,
viser at $M$ er stokastisk. Siden $S$ har strengt positive koefisienter vil $M$ også være regulær.
\newpage
\begin{tcolorbox}[colframe=white]
    \textbf{Oppgave V}
\end{tcolorbox}
$$M=
\begin{pmatrix}
    0 &   0   &   0             &   0   &   0.9\\
    0 &   0   &   0.45   &   0.9   &   0\\
    0 &   0   &   0             &   0   &   0\\
    0 &   0.9   &   0.45   &   0   &   0\\
    0.9 &   0   &   0             &   0   &   0
\end{pmatrix}+
\begin{pmatrix}
    0.02 &   0.02   &   0.02   &   0.02   &   0.02\\
    0.02 &   0.02   &   0.02   &   0.02   &   0.02\\
    0.02 &   0.02   &   0.02   &   0.02   &   0.02\\
    0.02 &   0.02   &   0.02   &   0.02   &   0.02\\
    0.02 &   0.02   &   0.02   &   0.02   &   0.02
\end{pmatrix}$$
$$=
\begin{pmatrix}
    0.02 &   0.02   &   0.02   &   0.02   &   0.92\\
    0.02 &   0.02   &   0.47   &   0.92   &   0.02\\
    0.02 &   0.02   &   0.02   &   0.02   &   0.02\\
    0.02 &   0.92   &   0.47   &   0.02   &   0.02\\
    0.92 &   0.02   &   0.02   &   0.02   &   0.02
\end{pmatrix}$$
\begin{tcolorbox}[colframe=white]
    \scriptsize
    \begin{minted}{matlab}
    >> B = M - I

    B =

       -0.9800    0.0200    0.0200    0.0200    0.9200
        0.0200   -0.9800    0.4700    0.9200    0.0200
        0.0200    0.0200   -0.9800    0.0200    0.0200
        0.0200    0.9200    0.4700   -0.9800    0.0200
        0.9200    0.0200    0.0200    0.0200   -0.9800
     
    >> null(B)
     
    ans =
     
        0.4011
        0.5816
        0.0401
        0.5816
        0.4011
     
    >> 
\end{minted}
\end{tcolorbox}
Score-vektoren blir da 
$
\begin{bmatrix}
    \frac{0.4011}{2.0056} \\[6pt] \frac{0.5816}{2.0056} \\[6pt] \frac{0.0401}{2.0056}\\[6pt]
    \frac{0.5816}{2.0056} \\[6pt] \frac{0.4011}{2.0056}
\end{bmatrix} =
\begin{bmatrix}
    0.20 \\ 0.29 \\ 0.02 \\
    0.29 \\ 0.20
\end{bmatrix}$
\begin{tcolorbox}[colframe=white]
    \textbf{Oppgave VI}
\end{tcolorbox}
Den første delen av koden setter koefisientene for de tre første kolonnene
\begin{tcolorbox}[colframe=white]
    \scriptsize
    \begin{minted}{matlab}
    function A = randlinkmatrix(n)
        A = round(rand(n,n));           // lager random matrise med bare nuller og enere
        for k=1:(n-1)                   // itererer over matrisen fra første kolonne til nest siste
            A(k,k) = 0;                 // setter diagonalelementet til 0
            if (A(:,k) == 0)            // hvis alle koefisientene i en kolonne er null:
                A(n,k) = 1;             //      → sett siste inngang i kolonnen til 1
            end                         //
            s = sum(A(:,k));            // s = summen av koefisientene i en kolonne
            A(:,k) = (1/s) * A(:,k);    // setter koefisientene i en kolonne slik at summen av de er 1    
\end{minted}
\end{tcolorbox}
Første del setter alle koefisientene bortsett fra de i siste kolonne til å være ett tall $x, \quad 0\leq x \leq 1$
slik at summen av koefisientene i en kolonne blir 1 (siste linje)
\newpage
Den siste delen av koden setter koefisientene for siste kolonne
\begin{tcolorbox}[colframe=white]
    \scriptsize
    \begin{minted}{matlab}
    A(n,n) = 0;                         // setter diagonalelementet til 0
    if (A(:,n) == 0)                    // hvis alle koefisientene i en kolonne er null:
        A(1,n) = 1;                     //      → sett første inngang i kolonnen til 1
    end                                 //
    s = sum(A(:,n));                    // s = summen av koefisientene i en kolonne
    A(:,n) = (1/s) * A(:,n);            // setter koefisientene i en kolonne slik at summen av de er 1 
    \end{minted}
\end{tcolorbox}
Denne delen av koden gjør nesten det samme som første del, eneste forskjellen er at hvis alle koefisientene
er null i kolonnen, blir første element satt til 1 og ikke siste. Dette må gjøres fordi en link-matrisen
kan bare ha 0 på diagonalen. Hvis en link-matrise har noe annet en nuller på diagonalen vil det tilsvare at
et dokument har en lenke til seg selv, noe definisjonen ikke tilater.
\begin{tcolorbox}[halign title=center, box align=top] 
    Denne funksjonen vil returnere en matrise med positive koefisienter som utgjør en sum på 1 i kolonnene, og ingen dokumenter har hyperlenker
    til seg selv. Denne funksjonen vil returnere en link-matrise
\end{tcolorbox}
\begin{tcolorbox}[colframe=white]
    \textbf{Oppgave VII}
\end{tcolorbox}
\begin{tcolorbox}[colframe=white]
    \scriptsize
    \begin{minted}{matlab}
    >> randlinkmatrix(5)

    ans =
        
             0         0         0         0    0.3333
        0.3333         0    0.5000         0         0
             0    0.3333         0    0.5000    0.3333
        0.3333    0.3333         0         0    0.3333
        0.3333    0.3333    0.5000    0.5000         0
        
    >> 
\end{minted}
\end{tcolorbox}
\begin{figure}[!htb]
    \centering
    \resizebox{0.28\textwidth}{!}{\input{oppgave7.pdf_tex}}
\end{figure}
\begin{tcolorbox}[colframe=white]
    \textbf{Oppgave VIII}
\end{tcolorbox}
\begin{tcolorbox}[colframe=white]
    \scriptsize
    \begin{minted}{matlab}
    function x = ranking(A)
        m = 0.1;
        n = size(A,1);
        S = ones(n) * 1/n;
        M = (1-m)*A + m*S;
        x = null(M - eye(n));
    end
\end{minted}
\end{tcolorbox}
\begin{tcolorbox}[colframe=white]
    \textbf{Oppgave IX}
\end{tcolorbox}
\begin{tcolorbox}[colframe=white]
    \scriptsize
    \begin{minted}{matlab}
    function x = rankingapprox(A,delta)
        m = 0.1;
        n = size(A,1);
        S = ones(n) * 1/n;
        M = (1-m)*A + m*S;
        xk_minus1 = ones(n,1) * 1/n;
        x = M * xk_minus1;
        k = 0;
        while max(x - xk_minus1) >= delta
            xk_minus1 = x;
            x = M * xk_minus1;
            k = k+1;
        end
        k
    end
\end{minted}
\end{tcolorbox}
\begin{tcolorbox}[colframe=white]
    \textbf{Oppgave X}
\end{tcolorbox}
\hspace*{0mm}\begin{minipage}[t]{.46\linewidth}  
\begin{tcolorbox}[box align=top, colframe=white]
    \scriptsize
    \begin{minted}{matlab}
    A =

             0         0    0.5000    1.0000
        0.3333         0         0         0
        0.3333    0.5000         0         0
        0.3333    0.5000    0.5000         0



    >> a = ranking(A)

        a =

            0.7050
            0.2586
            0.3749
            0.5436

    \end{minted}
\end{tcolorbox}
\end{minipage}
\hspace*{0mm}\begin{minipage}[t]{.46\linewidth}  
    \begin{tcolorbox}[box align=top, colframe=white]
        \scriptsize
        \begin{minted}{matlab}
    >> b = rankingapprox(A,0.01)

        b =

            0.3766
            0.1389
            0.1977
            0.2867

    >> a = 1/sum(a) * a

        a =

            0.3746
            0.1374
            0.1992
            0.2888

\end{minted}
\end{tcolorbox}
\end{minipage}\\
\entoure{\mintinline{text}{a}} og \entoure{\mintinline{text}{b}} er veldig like etter at \entoure{\mintinline{text}{a}}
er blitt gjort om til en score-vektor.
%
%
%
\end{document}